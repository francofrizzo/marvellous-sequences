\newcommand{\note}[1]{\textbf{\textcolor{red}{#1}}}

\section{Generalidades}

Sea $\alphabet = \lbrace0, 1\rbrace$ un alfabeto de dos caracteres.
Trabajaremos con secuencias finitas $s \in \alphabet^*$ formadas por estos dos
caracteres. Al referirnos a las posiciones dentro de una secuencia,
consideraremos que las mismas se numeran comenzando en $0$. Usaremos la
notación $s\substr{i}{j}$ para referirnos a la subsecuencia formada por las
posiciones de $s$ que van desde $i$ hasta $j$, inclusive.

\begin{defi}
  Definimos la función de rotación $\sigma: \alphabet^* \to \alphabet^*$ como
  \[ \sigma(\alpha_1\dots\alpha_{n-1}\alpha_n) =
    \alpha_n\alpha_1\dots\alpha_{n-1}. \]
\end{defi}

\begin{defi}
  Dada una secuencia $s \in \alphabet^*$, el \emph{collar} inducido por $s$
  (que notaremos $\neck{s} \in \alphabet^*\backslash\sigma$), es la clase de
  equivalencia de $s$ bajo $\sigma$. Un \emph{corte} de un collar $\neck{s}$
  es cualquier secuencia $s \in \neck{s}$.
\end{defi}

\begin{defi}

  Decimos que una palabra $w \in \alphabet^*$ \emph{aparece} en (o es una
  subsecuencia de) un collar $\neck{s}$ si existen $x, y \in \alphabet^*$ tales
  que $xwy = ss$, o equivalentemente, si existe una secuencia $s' \in \neck{s}$
  tal que $w$ es una subsecuencia de $s'$.
\end{defi}

\section{Definiciones y resultados previos}

\subsection{Secuencias de De Bruijn}

\begin{defi}
  Una secuencia $s \in \alphabet^*$ es una secuencia $n$-\emph{de De Bruijn}
  ($s \in \DB{n}$) si todos los patrones $w \in \alphabet^n$ (de longitud $n$),
  aparecen exactamente una vez en $\neck{s}$. En tal caso $\neck{s}$ es un
  collar de De Bruijn.
\end{defi}

\begin{obs}
  La longitud de una secuencia $n$-de De Bruijn es $2^n$.
\end{obs}

\begin{defi}
  Un \emph{grafo de De Bruijn} de orden $n$ es un grafo $G_n$ cuyos vértices
  son los elementos de $\alphabet^n$ y existe una arista entre $w_1$ y $w_2$ si
  y solo si ${w_1}\substr{1}{n-1} = {w_2}\substr{0}{n-2}$, es decir, si los
  últimos $n-1$ caracteres de $w_1$ coinciden con los primeros $n-1$ caracteres
  de $w_2$.
\end{defi}

\begin{obs}\ %
  \begin{itemize}
    \item Cada collar $n$-de De Bruijn se corresponde exactamente con un ciclo
          euleriano en el grafo de De Bruijn de orden $n - 1$.
    \item Cada collar $n$-de De Bruijn se corresponde exactamente con un ciclo
          hamiltoniano en el grafo de De Bruijn de orden $n$.
  \end{itemize}
\end{obs}

\begin{prop}
  Existen exactamente $2^{2^{n+1} - n}$ collares $n$-de De Bruijn.
\end{prop}

\begin{demosketch}
  Contar los ciclos eulerianos en $G_{n-1}$.
  En \cite{db} se prueba que la cantidad de dichos ciclos es exactamente
  $2^{2^{n-1} - n}$.
\end{demosketch}

\subsection{Secuencias maravillosas y maravillosas anidadas}

\subsubsection{Secuencias maravillosas}

\begin{defi}
  Una secuencia $s \in \alphabet^*$ es una secuencia $(n,m)$-\emph{maravillosa}
  ($s \in \M{n}{m}$) si todos los patrones $w \in \alphabet^n$ aparecen
  exactamente $m$ veces en $\neck{s}$. En tal caso, decimos que $\neck{s}$ es
  un collar maravilloso.
\end{defi}

\begin{obs}\ %
  \begin{itemize}
    \item La longitud de una secuencia $(n,m)$-maravillosa es $m \cdot 2^n$.
    \item Una secuencia es $n$-de De Bruijn si y solo si es
          $(n,1)$-maravillosa.
  \end{itemize}
\end{obs}

\begin{defi}
  Un \emph{multigrafo de De Bruijn} de orden $n$ y grado $m$ es un multigrafo
  $G_n^m$ que se obtiene a partir del grafo de De Bruijn de orden $n$
  replicando $m$ veces cada arista.
\end{defi}

\begin{obs}
  Las secuencias $(n,m)$-maravillosas se corresponden exactamente con los
  ciclos eulerianos (hamiltonianos) en el multigrafo de De Bruijn de orden $n -
    1$ ($n$) y grado $m$.
\end{obs}

\subsubsection{Secuencias maravillosas anidadas}

\begin{defi}
  Una secuencia $s \in \alphabet^*$ es una secuencia $(n,m)$-\emph{maravillosa
    anidada} ($s \in \NM{n}{m}$) si se cumple simultáneamente que:
  \begin{enumerate}
    \item $s$ es una secuencia $(n,m)$-maravillosa.
    \item Si $n > 1$, $s$ es la concatenación de dos secuencias
          $(n-1,m)$-maravillosas anidadas.
  \end{enumerate}

  Alternativamente, una secuencia es $(n,m)$-maravillosa anidada si, para todo
  $n'$ tal que $1 \leq n' \leq n$, toda subsecuencia de longitud
  $m \cdot 2^{n'}$ que comienza en una posición congruente a $0$ módulo
  $m \cdot 2^{n'}$ es $(n',m)$-maravillosa.
\end{defi}

\begin{obs}
  La propiedad de ser una secuencia maravillosa anidada no es invariante bajo
  rotaciones. Por lo tanto, la noción de collar maravilloso anidado no está
  bien definida.
\end{obs}

\note{Agregar la caracterización como ciclos anidados en multigrafos que hablamos por mail}

\subsection{Secuencias perfectas y perfectas anidadas}

\subsubsection{Secuencias perfectas}

\begin{defi}
  Una secuencia $s \in \alphabet^*$ es una secuencia $(n,m)$-\emph{perfecta}
  ($s \in \Pf{n}{m}$) si todos los patrones $w \in \alphabet^n$ aparecen
  exactamente $m$ veces en $\neck{s}$, y cada una de estas apariciones comienza
  en una posición distinta módulo $m$. En tal caso, decimos que $\neck{s}$
  es un collar perfecto.
\end{defi}

\begin{obs}
  Toda secuencia $(n,m)$-perfecta es también una secuencia $(n,m)$-maravillosa.
\end{obs}

\begin{defi}
  Un \emph{grafo astuto} de orden $n$ y grado $m$ es un grafo $\tilde{G}_n^m$
  cuyos vértices son los pares ordenados $(w, i) \in \alphabet^n \times \lbrace
    0,\dots,m-1 \rbrace$ y existe una arista entre $(w_1, i_1)$ y $(w_2, i_2)$
  si y solo si se cumplen simultáneamente:
  \begin{enumerate}
    \item ${w_1}\substr{1}{n-1} = {w_2}\substr{0}{n-2}$, es decir, los últimos
          $n-1$ caracteres de $w_1$ coinciden con los primeros $n-1$ caracteres
          de $w_2$, y además
    \item $v_1 + 1 \equiv v_2 \pmod{m}$.
  \end{enumerate}
\end{defi}

\begin{obs}
  Las secuencias $(n,m)$-perfectas se corresponden con los ciclos
  eulerianos (hamiltonianos) en el grafo astuto de orden $n-1$ (n) y grado $m$.

  En \cite{ALVAREZ201648} se brinda una fórmula para la cantidad de
  collares $(n,m)$-perfectos a partir de este hecho.
\end{obs}

\subsubsection{Secuencias perfectas anidadas}

\begin{defi}
  Una secuencia $s \in \alphabet^*$ es una secuencia $(n,m)$-\emph{perfecta
    anidada} ($s \in \NPf{n}{m}$) si se cumple simultáneamente que:
  \begin{enumerate}
    \item $s$ es una secuencia $(n,m)$-perfecta.
    \item Si $n > 1$, $s$ es la concatenación de dos secuencias
          $(n-1,m)$-perfectas anidadas.
  \end{enumerate}
\end{defi}

\begin{obs}\ %
  \begin{itemize}
    \item Toda secuencia $(n,m)$-perfecta anidada es también una secuencia
          $(n,m)$-maravillosa anidada.
    \item La propiedad de ser una secuencia perfecta anidada no es invariante
          bajo rotaciones, por lo que no está bien definida la noción de collar
          perfecto anidado.
  \end{itemize}
\end{obs}

\subsubsection{Secuencias afines}

\begin{defi}
  Sean $n$ y $m = 2^d$ tales que $n \leq m$.
  Una secuencia $(n,m)$-\emph{afín} es una secuencia $s \in
    \alphabet^{m \cdot 2^n}$ de la forma
  \[ s = (\mat{M} \cdot w_1)(\mat{M} \cdot w_2)\dots(\mat{M} \cdot w_{2^n}) \]
  donde:
  \begin{enumerate}
    \item $\mat{M} \in \mathbb{F}_2^{m \times m}$ es una matriz de la forma
          $\Sigma_{k_1,\dots,k_{m}}(\mat{P}_d)$, siendo:
          \begin{enumerate}
            \item $\mat{P}_d \in \mathbb{F}_2^{m \times m}$ la matriz de Pascal
                  con las columnas en orden inverso, definida como:
                  \[
                    \mat{P}_0 = \begin{bmatrix} 1 \end{bmatrix}; \qquad
                    \mat{P}_d = \begin{bmatrix}
                      \mat{P}_{d-1} & \mat{P}_{d-1} \\
                      0             & \mat{P}_{d-1} \\
                    \end{bmatrix} \text{ para } d > 0.
                  \]
            \item $k_m = 0$; \qquad $k_i \in \lbrace k_{i+1}, k_{i+1} + 1
                    \rbrace$.
            \item $\Sigma_{k_1,\dots,k_m}$ la transformación que consiste
                  en rotar $k_i$ veces hacia abajo la $i$-ésima columna de una
                  matriz, para cada $i = 1,\dots,m$.
          \end{enumerate}
    \item Cada $w_i$ es de la forma $w_i = u_i \oplus z$, siendo
          $u_1,\dots,u_{2^n} \in \alphabet^m$ la enumeración en orden
          lexicográfico de las primeras $2^n$ palabras de $\alphabet^m$, y $z
            \in \alphabet^m$ una palabra arbitraria.
  \end{enumerate}
\end{defi}

\begin{prop}
  Existen exactamente $2^{2m - 1}$ secuencias $(n,m)$-afines.
\end{prop}

\begin{demosketch}
  La construcción de una secuencia $(n,m)$-afín tiene dos parámetros libres:
  \begin{enumerate}
    \item Los $m - 1$ valores $k_1,\dots,k_{m - 1}$ que definen cómo se
          obtiene la matriz $\mat{M}$. Para cada uno de ellos existen $2$
          valores posibles, dando un total de $2^{m - 1}$ posibilidades.
    \item La palabra arbitraria $z$, para la cual existen $2^m$
          posibilidades.
  \end{enumerate}
  Esto quiere decir que existen $2^{m - 1}\cdot2^m = 2^{2m - 1}$
  formas distintas de elegir los parámetros con los cuales construir una
  secuencia $(n,m)$-afín. Resta ver que para cada combinación posible de
  parámetros, la secuencia obtenida es única. La demostración de este hecho se
  sigue de que las matrices empleadas en el procedimiento son inversibles,
  y puede leerse en \cite{BECHER2019101403}.
\end{demosketch}

\begin{teo}
  Para $n$ y $m = 2^d$ tales que $n \leq m$, las secuencias $(n,m)$-afines se
  corresponden exactamente con las secuencias $(n,m)$-perfectas anidadas.
\end{teo}

\begin{demosketch}
  La demostración puede verse en \cite{BECHER2019101403}. En primer lugar se
  prueba que las secuencias afines son perfectas anidadas, analizando
  meticulosamente las características de la construcción, y luego se muestra
  que todas las secuencias perfecta son afines a través de un argumento de
  cardinalidad.
\end{demosketch}

\begin{coro}
  Para $n$ y $m = 2^d$ tales que $n \leq m$, existen exactamente $2^{2m - 1}$
  secuencias $(n,m)$-perfectas anidadas.
\end{coro}

\section{Resultados}

Nuestro primer resultado es:

\begin{teo}
  \label{teo:existencia}
  Sea $d > 0$. Existen secuencias $(d + 1, 2^d)$-maravillosas anidadas que no son perfectas anidadas.
\end{teo}

Ejemplos: La secuencia [inserte ejemplo aquí] es $(3, 4)$-maravillosa anidada pero no es perfecta anidada.
La secuencia [inserte ejemplo aquí] es $(4, 8)$-maravillosa anidada pero no es perfecta anidada.

  [Nuestro ideal: ensayar esto para $n = m$]

  [Agregamos la tabla experimental de cantidades de maravillosas/perfectas anidadas para cada $n, m$]

  [Agregamos la tabla experimental de cantidades de maravillosas/perfectas anidadas concatenando De Bruijn para cada $n, m$]

\begin{prop}
  No existen secuencias $(n,m)$-maravillosas anidadas con $n > 2m$.
\end{prop}

\begin{demo}
  En primer lugar, debemos notar que basta probar el resultado para $n = 2m +
    1$, ya que cualquier secuencia $(n',m)$-maravillosa anidada con $n' > 2m +
    1$ consiste en la concatenación de varias secuencias
  $(2m+1,m)$-maravillosas anidadas.

  \note{Fijemos m.}
  Razonando por el absurdo, sea $n = 2m + 1$ y supongamos que existe una
  secuencia $s \in \NM{n}{m}$. Por definición, el patrón $0^{2m+1}$ debe
  aparecer $m$ veces en $\neck{s}$. Sea $i$ la posición inicial en $s$, tomando
  módulo $2m$, de la primera aparición de dicho patrón.
  \note{no entiendo aqui ``tomando modulo 2m''. Yo lo sacaría.}

  Recordemos que $s$ consiste en una concatenación de $m$-átomos, cada uno de
  los cuales tiene longitud $2m$. Los $2m+1$ ceros del patrón mencionado
  necesariamente abarcarán más de un átomo, con los primeros $2m - i$ ceros
  ubicados dentro de un átomo y los $i + 1$ ceros restantes en el átomo
  siguiente. Notar que, si $i < m$, entonces $2m - i > m$, por lo que hay más
  de $m$ ceros en el primer átomo; si, por el contrario, $i \geq m$, entonces
  $i + 1 > m$, por lo que hay más de $m$ ceros en el segundo átomo. Ninguno
  de los dos casos es posible, ya que la cantidad de ceros en un $m$-átomo es
  exactamente $m$. \qed
\end{demo}

\subsection{Métodos de construcción}

\begin{defi}
  Llamamos $m$-\emph{átomos} a las secuencias $\omega \in \alphabet^{2m}$ que
  son $(1,m)$-maravillosas.
\end{defi}

\begin{obs}
  Toda secuencia $(n,m)$-maravillosa es una concatenación de $2^{n-1}$
  $m$-átomos.
\end{obs}

\begin{prop}
  Existen exactamente $\binom{2m}{m}$ $m$-átomos distintos, correspondientes a
  las formas de elegir las $m$ posiciones que serán ocupadas por el símbolo
  $0$.
\end{prop}

\begin{demo}
  Una secuencia es un $m$-átomo si y solo si es una permutación de la palabra
  $0^m1^m$.

  Por definición, en un $m$-átomo todos los patrones de longitud $1$, es
  decir, los caracteres $0$ y $1$ deben aparecer exactamente $m$ veces. \qed
\end{demo}

\note{No tenemos definido lo que es una permutacion de una palabra.
  Permutacion de una secuencia sí podes usar, porque viene del significado usual en matemática.
  Tal vez sea mejor así: es una secuencia que
  contiene exactamente m unos y m ceros, encualquier orden.}

\begin{defi}
  Una secuencia $s$ es $(n,m)$-\emph{autosimilar} si es de la forma
  $\omega^{2^{n-1}}$, donde $\omega$ es un $m$-átomo. Es decir, si consiste
  en $2^{n-1}$ repeticiones de un mismo $m$-átomo.
\end{defi}

\begin{obs}
  \label{lema:autosimilar-cannot-be-np}
  Una secuencia $(n,m)$-autosimilar con $n > 1$ no puede ser $(n,m)$-perfecta
  anidada.
\end{obs}

\begin{demo}
  Dado que la secuencia se construye repitiendo % al menos dos veces
  un mismo
  átomo de longitud $2m$, el patrón de longitud $n$ que aparece en la posición
  $0$ de la secuencia aparece nuevamente en la posición $2m$. Esto contradice
  la definición de secuencia perfecta, ya que cada aparición del patrón debería
  comenzar en una posición distinta módulo $m$. \qed
\end{demo}

\begin{lema}
  Sea $\omega \in \alphabet^{2^n}$ una secuencia $n$-de De Bruijn. Para todo
  $n'$ tal que $1 \leq n' \leq n$, la secuencia $\omega$ es
  $(n',2^{n-n'})$-maravillosa.
\end{lema}

\note{
  Generalizar este lema para hablar de maravillosas en vez de De Bruijn.
}

\begin{demo}
  Consideremos un patrón $w$ cualquiera de longitud $n'$. Para extender $w$ a
  un patrón de longitud $n$ debe agregársele un sufijo de longitud $n-n'$, por
  lo que existen $2^{n-n'}$ formas distintas de hacerlo. Todas
  %estas extensiones posibles 
  \note{los patrones}
  aparecen exactamente una vez en $\omega$, ya que esta es
  una secuencia $n$-de De Bruijn. Luego, $w$ aparece exactamente $2^{n-n'}$
  veces en $\omega$, cada vez como prefijo de un patrón de longitud $n$
  distinto. Por lo tanto, $s$ es una secuencia $(n',2^{n-n'})$-maravillosa.
  \qed
\end{demo}

\begin{coro}
  Una secuencia $n$-de De Bruijn es un $(2^{n-1})$-átomo.
\end{coro}

\begin{demo}
  Resulta del lema anterior, tomando $n' = 1$. \qed
\end{demo}


\begin{lema}
  Sea $\omega$ una secuencia $n$-de De Bruijn.
  Para todo $n'$ tal que $n' \leq n$ y para todo $d \geq 0$, la
  secuencia $\omega^{2^d}$ es $(n',2^{n-n'+d})$-maravillosa \note{anidada}.

  \note{Generalizar este lema para considerar las maravillosas.}
\end{lema}

\begin{demo}
  \note{Esta demo resulta de unificar dos demos; pulir.}
  El caso $d = 0$ es exactamente lo probado en el lema anterior.
  Consideremos $d > 0$ y supongamos que la propiedad vale para $d - 1$.
  Sea $s' = \omega^{2^{d-1}}$; podemos ver que $s = s's'$.

  Consideremos un patrón $w$ cualquiera de longitud $n'$. Por hipótesis
  inductiva, $s'$ es una secuencia $(n',2^{n-n'+d-1})$-maravillosa. Luego, $w$
  debe aparecer $2^{n-n'+d-1}$ veces en $\neck{s'}$.
  Esto quiere decir que la cantidad de apariciones de $w$ en $\neck{s} =
    \neck{s's'}$ es exactamente el doble: $2 \cdot 2^{n-n'+d-1} = 2^{n-n'+d}$.
  Por lo tanto, $s$ es una secuencia $(n',2^{n-n'+d})$-maravillosa. \qed

  Consideremos $n'$ tal que $1 \leq n' \leq n$. Basta ver que todas las
  subsecuencias de $s$ con longitud $m \cdot 2^{n'}$ que comienzan en
  posiciones congruentes a $0$ módulo $m \cdot 2^{n'}$ son
  $(n',m)$-maravillosas. Observemos que $m \cdot 2^{n'} = 2^{n+n'-1} = 2^n
    \cdot 2^{n'-1}$ es múltiplo de $2^n$, la longitud de $\omega$.
  Luego, estas subsecuencias son todas de la forma $\omega^{2^{n'-1}}$.
  Aplicando el lema anterior, con $d = n'-1$, tenemos que $\omega^{2^d}$
  es $(n',2^{n-1})$ maravillosa, que es lo que queríamos demostrar. \qed
\end{demo}
\note{Si sacas m del enunciado del lema anterior hay que cambiar este tambien}

% Esto ahora quedó en el lema anterior
% \begin{prop}
%   \label{prop:db-concatenation-is-nm}
%   Sea $\omega \in \alphabet^{2^n}$ una secuencia $n$-de De Bruijn.
%   Sea $m = 2^{n-1}$. La secuencia $s = \omega^{m}$ es $(n,m)$-maravillosa
%   anidada.
% \end{prop}

\note{Llegamos hasta acá. Estamos listos para dar el método de construcción.}

\begin{obs}
  En la Proposición \ref{prop:db-concatenation-is-nm}, notar que $\omega$ es un
  $m$-átomo. Por lo tanto, la secuencia $s = \omega^m$ es $(n,m)$-autosimilar.
\end{obs}

\note{Segun lo que hablamos, este enuciado será acerca de cortes de collares de Bruijn}
\begin{lema}
  \label{lema:db-sequence-count}
  Existen exactamente $2^{2^{n-1}}$ secuencias $n$-de De Bruijn.
\end{lema}

\begin{demo}
  Sea $\neck{s}$ un collar $n$-de De Bruijn, y sea $s$ un corte de dicho
  collar. Para cada $i = 0,\dots,2^n-1$, el resultado de aplicar $\sigma^i(s)$
  debe ser una secuencia distinta, ya que de existir $j \neq j'$ tales que
  $\sigma^j(s) = \sigma^{j'}(s)$, el patrón de longitud $n$ que comienza en la
  posición $j$ de $s$ aparecería de nuevo en la posición $j$, lo cual no es
  posible. Luego, por cada collar $n$-de De Bruijn, se tienen $2^n$ secuencias
  $n$-de De Bruijn distintas. Desde luego, las secuencias que corresponden a
  collares distintos son también distintas.
  \note{Avisa que  estas usando el resultado de de Bruijn que ya mencionaste en la seccion dedicada a de Bruijn}.
  Por lo tanto, se tiene un total de
  $2^n \cdot 2^{2^{n-1}-n} = 2^{2^{n-1}}$ secuencias $n$-de De Bruijn. \qed
\end{demo}


\note{Creo que menos variables es mejor. Que tal asi?\\
  Sea $d > 0$.
  Existen al menos $2^{2^d}$
  secuencias $(d+1, 2^d)$-maravillosas anidadas que no son perfectas anidadas.
}


\begin{demo}[Teorema \ref{teo:existencia}]
  Por el Lema \ref{lema:db-sequence-count}, existen $2^{2^{n-1}} = 2^m$ formas
  distintas de elegir la secuencia $\omega$ que aparece en la Proposición
  \ref{prop:db-concatenation-is-nm}. Para cada una de ellas se construye una
  secuencia $s$ que es $(n, m)$-maravillosa anidada. Como además las
  secuencias obtenidas son $(n,m)$-autosimilares y $n > 1$, aplicando el Lema
  \ref{lema:autosimilar-cannot-be-np}, se tiene que ninguna de las secuencias
  es perfecta anidada. \qed
  \note{La demostración está bien.
    Propongo algo así:
    \\
    Para cada corte de collar de Bruijn construimos
    una secuencia $(d+1, 2^d)$-maravillosa anidada.
    El Lema~\ref{lema:autosimilar-cannot-be-np} nos asegura
    que no  es perfecta anidada.
    Hay $2^{2^d}$ cortes de secuencias de Bruijn distintos.
  }
\end{demo}

\begin{defi}
  Consideremos el grafo de De Bruijn de orden $n$, $G_n$. Sea $w \in
    \alphabet^n$ un nodo \note{vertice} de dicho grafo
  y sea $k$ entre $0$ y $n$
  %$0 \leq k \leq n$.
  Decimos que $w' \in \alphabet^n$ es un \emph{$k$-sucesor} de $w$ si en $G_n$
  existe un camino de longitud $k$ que va de $w$ a $w'$.
  Análogamente, decimos que $w'$ es un \emph{$k$-antecesor} de $w$ si en $G_n$
  existe un camino de longitud $k$ que va de $w'$ a $w$.

  Notamos $\Succ{k}(w)$ y $\Pred{k}(w)$ a los conjuntos de $k$-sucesores y
  $k$-predecesores de $w$, respectivamente.
\end{defi}

\begin{obs}\ %
  \begin{enumerate}
    \item La cantidad de $k$-sucesores (antecesores) de todo nodo es
          exactamente $2^k$.
    \item Para todo nodo, $\Succ{n}(w) = \Pred{n}(w) = \alphabet^n$.
  \end{enumerate}
\end{obs}

\begin{defi}
  Sean $w,w',u,u' \in \alphabet^n$. Decimos que $(w,w')$ y $(u,u')$ son
  \emph{pares cruzados} si los $n$ patrones de longitud $n + 1$ que aparecen en
  $ww'$ y $uu'$ son los mismos que aparecen en $wu$ y $w'u'$.
\end{defi}

\begin{obs}
  La relación entre pares cruzados es reflexiva y simétrica, pero no
  transitiva.
\end{obs}

\begin{prop}
  Sean $w, w' \in \alphabet^n$. Existen exactamente $(n + 1) \cdot 2^n - n
    \cdot 2^{n - 1}$ pares $(u, u')$ tales que $(w, w')$ y $(u, u')$ son pares
  cruzados.
\end{prop}

\begin{demo}
  Sean $w_0 = w$, $w_n = w'$ y $w_0, w_1, \dots, w_n$ el camino de
  $n$ aristas ($n + 1$ nodos) en $G_n$ que va de $w_0$ a $w_n$.

  A continuación, definimos
  \[ A_0 = \lbrace w_0 \rbrace, \qquad
    B_0 = \alphabet^n, \]
  y para cada $i = 1, \dots, n$:
  \[ A_i = \Pred{i}(w_i) \,\backslash\, \Pred{i - 1}(w_{i - 1}), \qquad
    B_i = \Succ{n - i}(w_i). \]

  Afirmamos que el conjunto de pares cruzados con $(w_0, w_n)$ es
  \[ C = \bigcup_{i = 0}^n A_i \times B_i. \]

  La cantidad de elementos de $C$ es:
  \begin{align*}
    \sum_{i = 0}^n \vert A_i \vert \times \vert B_i \vert
     & = \sum_{i = 1}^n \vert A_i \vert \times \vert B_i \vert + \vert A_0
    \vert \times \vert B_0 \vert                                           \\
     & = \sum_{i = 1}^n \left( 2^{i} - 2^{i - 1} \right) \cdot 2^{n - i} +
    2^n                                                                    \\
     & = \sum_{i = 1}^n \left( 2^n - 2^{n - 1} \right) + 2^n               \\
     & = (n + 1) \cdot 2^n - n \cdot 2^{n - 1}.
  \end{align*}
\end{demo}
