\documentclass[a4paper,11pt]{article}

\usepackage{a4wide}
\usepackage[spanish]{babel}
\usepackage[utf8]{inputenc}
\usepackage{url}

\usepackage{amsmath, amscd, amssymb, amsthm, latexsym}





\begin{document}


\newtheorem*{problema}{Problema}

\begin{center}

{\Large\bf Propuesta de Tesis de Licenciatura\\
 en Ciencias de la Computación}
\medskip
\\
Departamento de Computaci\'on\\
 Facultad de Ciencias Exactas y Naturales\\
Universidad de Buenos Aires
\end{center}
\bigskip

\noindent
\begin{tabular}{ll}
Tema:   &{\large \bf Secuencias maravillosas}\\
Alumno: &Franco FRIZZO\\
Email:  &francofrizzo@gmail.com\\
%LU: & \\
Directores:&  Ver\'onica Becher \\
E-mail: &vbecher@dc.uba.ar \\
Fecha:& 28 Septiembre   2020\\
\end{tabular}
\bigskip


\section*{\bf Problema}
Una secuencia $s$ de símbolos de $\{0,1\}$  es {\em maravillosa} de orden $(n,m)$
si todas las secuencias 
 de longitud $n$ ocurren exactamente  $n$ veces en
 la palabra circular definida por $s$.
Una  secuencia es {\em maravillosa anidada} de orden $(n,m)$
si es maravillosa  de orden $(n,m)$
y, cuando  $n>1$, es la concatenación  dos secuencias
 maravillosas  anidadas de orden $(n-1,m)$.

Se sabe que para todo $(n,m)$
 existen las secuencias maravillosas anidadas que cumplen que 
todas las secuencias de longitud $m$ ocurren en distintas posiciones módulo $m$.
¿Es necesaria esta condición de ocurrencia en posiciones 
que son distintas módulo $m$?
Es decir, ¿Existen las secuencias maravillosas anidadas de orden $(n,m)$ 
sin la restricción de que las ocurrencias
sean en distintas posiciones módulo $m$?

\section*{Antecedentes}

Consideremos el alfabeto de dos símbolos, $\{0,1\}$.
  Una secuencia $s$ de símbolos de $\{0,1\}$ es una secuencia de  Bruijn
de orden $n$ si todos las secuencias  de longitud $n$,
  aparecen exactamente una vez en la palabra circular definida por $s$.
Las secuencias de Bruijn de orden $n$ tienen longitud $2^n$.
Nicolaas~Govert de~Bruijn \cite{db} dio esta definición al mismo tiempo que contó 
las caracterizó como etiquetas de ciclos Eulerianos en los llamados grafos de Bruijn,
y contó cuantas hay.


Recientemente  Alvarez, Becher, Ferrari y  Yuhjtman \cite{perfectos}
presentaron una generalización de las secuencias de Bruijn,
que se llaman {\em secuencias perfectas}.
  Una secuencia $s $ de símbolos de $\{0,1\}$  es una secuencia
 \emph{perfecta} de orden $(n,m)$  si todas las secuencias 
 de longitud $n$ ocurren exactamente  $n$ veces en
 la palabra circular definida por $s$  en distintas  posiciones módulo~$m$.
Las secuencias perfectas de orden $(n,m)$ tiene longitud  $m2^n$.
En \cite{perfectos} las secuencias perfectas están caracterizadas como las etiquetas de 
ciclos Eulerianos en los llamados grafos astutos, y 
hay una fórmula para la cantidad de secuencias perfectas de orden $(n,m)$.


Becher y Carton \cite{perfectosanidados} definieron 
 las {\em secuencias perfectas anidadas} de orden $(n,m)$
como las que son  perfectas de orden $(n,m)$
y, cuando  $n>1$, son la concatenación  dos secuencias
 perfectas anidadas de orden $(n-1,m)$.
En \cite{perfectosanidados} las secuencias perfectas anidadas están caracterizadas 
como secuencias , que se obtienen a partir de una  transformación
lineal que  se optiene a partir del triángulo de Pascal. También 
 este trabajo cuenta la cantidad de secuencias perfectas anidadas
$(n,m) $  cuando $m$ es potencia de~$2$.

Becher y Carton definieron las  {\em secuencias maravillosas anidadas} 
de orden $(n,m)$ soltando la restricción de que las $m$ 
ocurrencias debían ocurrir  en 
posiciones distintas módulo~$m$, y dejaron abierta la pregunta de si existen o no las 
{\em secuencias maravillosas anidadas} que no son {\em perfectas anidadas}.


\section*{Objetivo}
Nos proponemos responder la pregunta formulada por Becher y Carton 
 de si existen secuencias maravillosas anidadas que no son perfectas anidadas.
En caso de responder positivamente nos proponemos dar algoritmos de construcción de estas secuencias
maravillosas.  Si fuera posible nos proponemos contar la 
cantidad de secuencias maravillosas anidadas de orden $(n,m)$.






\begin{thebibliography}{9}

\bibitem{perfectos}
Nicolás Alvarez, Verónica Becher, Pablo~A. Ferrari, and Sergio~A. Yuhjtman.
\newblock Perfect necklaces.
\newblock {\em Advances in Applied Mathematics}, 80:48 -- 61, 2016.

\bibitem{perfectosanidados}
Verónica Becher and Olivier Carton.
\newblock Normal numbers and nested perfect necklaces.
\newblock {\em Journal of Complexity}, 54:101403, 2019.

\bibitem{db}
Nicolaas~Govert de~Bruijn.
\newblock A combinatorial problem.
\newblock {\em Proc. Koninklijke Nederlandse Academie van Wetenschappen},
  49:758--764, 1946.

\end{thebibliography}
\end{document}:


 