\documentclass[11pt]{article}
\usepackage[a4paper, margin=2.25cm, headsep=1em]{geometry}

\usepackage[spanish]{babel}
\spanishlcroman
\usepackage[utf8]{inputenc}
\usepackage[T1]{fontenc}
\usepackage{lmodern}
\usepackage[letterspace=25]{microtype}

\usepackage{amsmath,amssymb,amsthm}
\usepackage{bm}
\usepackage{cases}
\usepackage{xfrac}
\usepackage{color}
\usepackage{framed}
\usepackage{fancyhdr}

\usepackage{enumitem}
\setlist[enumerate,1]{label=(\alph*)}
\setlist[enumerate,2]{label=\roman*.}

\usepackage{float}
\usepackage[center]{caption}

\usepackage{cite}

\newcommand\myauthor{Franco Frizzo}
\newcommand\mytitle{Tesis de licenciatura}
\newcommand\mydate{\today}

\usepackage[pdfauthor={\myauthor},
	    pdftitle={\mytitle},
	    hidelinks]{hyperref}

\setlength{\parskip}{.5em}
\renewcommand{\baselinestretch}{1.05}
\setlength{\headsep}{1.7em}
\setlist[enumerate]{itemsep=.1em, topsep=0em}
\setlist[itemize]{itemsep=.1em, topsep=0em}

\pagestyle{fancy}
\rhead{\MakeUppercase{\footnotesize{\textls{\myauthor}}}}
\lhead{\MakeUppercase{\footnotesize{\textls{\mytitle}}}}

\DeclareTextFontCommand{\emph}{\bfseries}

\theoremstyle{plain}
\newtheorem{teo}{Teorema}
\newtheorem{prop}[teo]{Proposición}
\newtheorem{coro}[teo]{Corolario}
\newtheorem{lema}[teo]{Lema}

\theoremstyle{definition}
\newtheorem{defi}[teo]{Definición}

\theoremstyle{remark}
\newtheorem*{obs}{Observación}
\newtheorem*{demo}{Demostración}
\newtheorem*{demosketch}{Idea de la demostración}

\newcommand{\note}[1]{\textbf{\textcolor{red}{#1}}}

\newcommand{\alphabet}{\ensuremath{\mathcal{A}}}

\newcommand{\neck}[1]{\left\vert#1\right\vert}
\newcommand{\substr}[2]{_{[#1\,..\,#2]}}

\newcommand{\mat}[1]{\mathbf{#1}}

\newcommand{\Succ}[1]{\ensuremath{\text{S}_{#1}}}
\newcommand{\Pred}[1]{\ensuremath{\text{P}_{#1}}}

\newcommand{\DB}[1]{\ensuremath{\text{DB}_{#1}}}
\newcommand{\M}[2]{\ensuremath{\text{M}_{#1}^{#2}}}
\newcommand{\NM}[2]{\ensuremath{\text{NM}_{#1}^{#2}}}
\newcommand{\Pf}[2]{\ensuremath{\text{P}_{#1}^{#2}}}
\newcommand{\NPf}[2]{\ensuremath{\text{NP}_{#1}^{#2}}}


\begin{document}

\title{\mytitle}
\author{\myauthor}
\date{\mydate}

\maketitle
\tableofcontents

\section{Introducción}

\section{Preliminares}

Definir secuencia maravillosa.

Dar conteo de secuencias maravillosas.
Son $\binom{2m}{m}^{2^{n-1}}$.

Definir secuencia maravillosa anidada.

Definir secuencia perfecta anidada.

Definir $m$-átomo.

\section{Métodos de construcción}

\subsection{Secuencias autosimilares}

En este apartado, presentaremos un método de construcción que permite generar
secuencias $(n,m)$-maravillosas anidadas que no son perfectas anidadas, siempre
y cuando se cumpla que $m = 2^{n - 1}$.
Este método fue motivado por el estudio de una subfamilia de las secuencias
maravillosas anidadas, que denominamos \emph{secuencias autosimilares}.

\begin{defi}
	Una secuencia $x$ es $(n,m)$-\emph{autosimilar} si consiste en $2^{n-1}$
	repeticiones de una misma secuencia de longitud $2m$.
\end{defi}

\begin{example}
	La secuencia \[ 01110110\ 01110110\ 01110110\ 01110110 \] es
	$(3,4)$-autosimilar.
	La secuencia \[ 110000101011 \ 110000101011 \] es $(2,6)$-autosimilar; además,
	es $(2,6)$-maravillosa anidada.
\end{example}

\begin{prop} \label{lemma:autosimilar-cannot-be-np}
	Una secuencia $(n,m)$-autosimilar con $n > 1$ no puede ser $(n,m)$-perfecta.
\end{prop}

\begin{demo}
	Dado que la secuencia se construye repitiendo al menos dos veces una misma
	secuencia de longitud $2m$, la palabra de longitud $n$ que aparece en la
	posición $0$ de la secuencia aparece nuevamente en la posición $2m$.
	Esto contradice la definición de secuencia perfecta, ya que cada aparición de
	dicha palabra debería comenzar en una posición distinta módulo $m$.
	\qed
\end{demo}

Nos interesa estudiar las secuencias $(n,m)$-maravillosas anidadas que, además,
son autosimilares.
Cabe señalar que las mismas están conformadas por un único $m$-átomo repetido
$2^{n-1}$ veces.
La proposición que enunciaremos a continuación nos permite caracterizarlas a
partir de una propiedad verificable los átomos que les dan origen.
Este resultado nos permitirá determinar para qué valores de $(n,m)$ existen
este tipo de secuencias, cuál es su cantidad, y presentar un método para su
construcción.

\begin{prop}
	\label{prop:autosimilar-nm-characterization}
	Sea $x = \omega^{2^{n-1}}$ una secuencia $(n,m)$-autosimilar.
	$x$ es $(n,m)$-maravillosa anidada si y solo si $2^{n-1} \vert m$ y
	$\omega$
	es una secuencia $\left( n, \frac{m}{2^{n-1}} \right)$-maravillosa.

	\note{Quizás esto queda mejor enunciado escribiendo $m = k \cdot
			2^{n-1}$, como está en el Corolario
		\ref{coro:autosimilar-nm-count}.}
\end{prop}

\begin{example}\ %
	\begin{itemize}
		\item El $4$-átomo $01011100$ es $(1,4)$, $(2,2)$ y
		      $(3,1)$-maravilloso.
		      Por lo tanto, las siguientes tres secuencias son $(1,4)$, $(2,4)$ y
		      $(3,4)$-maravillosas anidadas, respectivamente.
		      \[ 01011100; \qquad 01011100\ 01011100; \qquad 01011100\
			      01011100\ 01011100\ 01011100.
		      \]
		\item El $4$-átomo $10001101$ es $(1,4)$ y $(2,2)$-maravilloso,
		      pero no $(3,1)$-maravilloso.
		      Por lo tanto, las siguientes dos secuencias son $(3,1)$ y $(3,2)$-maravillosas
		      anidadas, respectivamente.
		      \[ 10001101; \qquad 10001101\ 10001101,
		      \]
		      pero la secuencia
		      \[ 10001101\ 10001101\ 10001101\
			      10001101 \]
		      no es $(3,4)$-maravillosa anidada.
	\end{itemize}
\end{example}

Los siguientes dos lemas se presentan a modo de resultados intermedios para
demostrar la proposición precendente.

\begin{lema} \label{lemma:marvellous-for-smaller-n}
	Sea $x$ una secuencia $(n,m)$-mavarillosa.
	Para todo $k$ tal que $0 \leq k < n$, la secuencia $x$ es $(n - k,2^{k}
		m)$-maravillosa.
\end{lema}

\begin{example}
	La secuencia $(3,4)$-maravillosa \[ 00001111\ 00101011\ 00101101\ 11100010, \]
	es también $(2,8)$ y $(1,16)$-maravillosa.
\end{example}

\begin{demo}
	Consideremos una palabra $w$ cualquiera de longitud $n - k$.
	Para extender $w$ a una palabra de longitud $n$ debe agregársele un sufijo de
	longitud $k$, por lo que existen $2^{k}$ formas distintas de hacerlo.
	Todas estas palabras que resultan de extender $w$ aparecen exactamente $m$
	veces en $x$, ya que esta es una secuencia $(n,m)$-mavarillosa.
	Luego, $w$ aparece exactamente $2^{k}m$ veces en $x$, $m$ veces como prefijo de
	una palabra de longitud $n$ distinto.
	Por lo tanto, $x$ es una secuencia $(n - k,2^{k}m)$-maravillosa.
	\qed
\end{demo}

\begin{lema}
	\label{lemma:marvellous-for-concatenation}
	Una secuencia $x$ es $(n,m)$-mavarillosa si y solo si para todo $d > 0$, la
	secuencia $x^d$ es $(n,dm)$-maravillosa.

	\note{No estoy seguro de si poner este lema.
		Sirve para simplificar un poco la demo siguiente.
	}
\end{lema}

Habiendo probado estos dos resultados, estamos en condiciones de presentar una
demostración para la Proposición \ref{prop:autosimilar-nm-characterization}.

\begin{demo}[Proposición \ref{prop:autosimilar-nm-characterization}]
	Si partimos del supuesto de que $2^{n-1}$ no divide a $m$, o bien que
	$\omega$ no es una secuencia $\left( n, \frac{m}{2^{n-1}}
		\right)$-maravillosa,
	podemos concluir que existe una palabra $w$ de longitud $n$ cuya
	cantidad de
	apariciones en $\omega$, mirada circularmente, es diferente de
	$\frac{m}{2^{n-1}}$.
	Por lo tanto, la cantidad de apariciones de $w$ en $x = \omega^{2^{n-1}}$,
	mirada también circularmente, es diferente de $m$.
	Es decir, en este caso, $x$ no puede ser una secuencia $(n,m)$-maravillosa
	anidada.

	\note{En el párrafo anterior podría usar el Lema
		\ref{lemma:marvellous-for-concatenation} y queda más simple.}

	Para demostrar la implicación opuesta, supongamos que $2^{n-1}$ y $\omega$ es
	una secuencia $\left( n, \frac{m}{2^{n-1}} \right)$-maravillosa.
	Demostraremos que para todo $n'$ tal que $1 \leq n' \leq n$, la secuencia
	$\omega^{2^{n'-1}}$ es $(n',m)$-maravillosa anidada.
	En el caso particular $n' = n$, esto equivale a decir que $x$ es una secuencia
	$(n,m)$-maravillosa anidada.

	En el caso $n' = 1$, el resultado se sigue del Lema
	\ref{lemma:marvellous-for-smaller-n} que, tomando $k = n-1$, nos permite
	afirmar que $\omega$ es una secuencia $(1,m)$-maravillosa.

	Si $n'$ es tal que $1 < n' \leq n$, supongamos probado el resultado para
	$n'-1$.
	Esto quiere decir que, siendo $y = \omega^{2^{n' - 2}}$, $y$ es una secuencia
	$(n'-1, m)$-maravillosa anidada.
	Para probar que $\omega^{2^{n' - 1}} = yy$ es $(n',m)$-maravillosa anidada,
	solo resta ver que es $(n',m)$-maravillosa, ya que es la concatenación de dos
	secuencias $(n'-1, m)$-maravillosas anidadas.

	Empleando otra vez el Lema \ref{lemma:marvellous-for-smaller-n}, esta vez para
	$k = n - n'$, tenemos que $\omega$ es una secuencia $\left( n', \frac{m}{2^{n'
				- 1}} \right)$.
	Esto significa que la cantidad de apariciones de cada palabra de longitud $n'$
	en $\omega$, mirada circularmente, es $\frac{m}{2^{n' - 1}}$; y por
	consiguiente, es $m$ en $\omega^{2^{n'-1}}$, por lo que esta última secuencia
	es $(n',m)$-maravillosa anidada.

	\note{El párrafo anterior también se puede simplificar con el Lema
		\ref{lemma:marvellous-for-concatenation}.}
	\qed
\end{demo}

Dado que conocemos la cantidad de secuencias maravillosas que existen para
cualesquiera valores de $n$ y $m$, podemos enunciar el siguiente corolario.

\begin{coro} \label{coro:autosimilar-nm-count}
	Sean $n, m > 0$.
	Si existe $k > 0$ tal que $m = k \cdot 2^{n-1}$, existen
	$\binom{2k}{k}^{2^{n-1}}$ secuencias $(n,m)$-maravillosas anidadas que son
	autosimilares.
	De lo contrario, no existe ninguna tal secuencia.
\end{coro}

\begin{center}
	\begin{tabular}{ c | c | c | c | c | c | c | c | c | c }
		  & 1 & 2 & 3  & 4  & 5   & 6   & 7    & 8     & 9     \\
		\hline
		1 & 2 & 6 & 20 & 70 & 252 & 924 & 3432 & 12870 & 48620
		\\ \hline
		2 & 0 & 4 & 0  & 36 & 0   & 400 & 0    & 4900  & 0     \\
		\hline
		3 & 0 & 0 & 0  & 16 & 0   & 0   & 0    & 1296  & 0     \\
		\hline
		4 & 0 & 0 & 0  & 0  & 0   & 0   & 0    & 256   & 0     \\
		\hline
		5 & 0 & 0 & 0  & 0  & 0   & 0   & 0    & 0     & 0
	\end{tabular}
\end{center}

\bibliography{references}{}
\bibliographystyle{plain}

\end{document}
