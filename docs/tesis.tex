\documentclass[11pt]{article}
\usepackage[a4paper, margin=2.25cm, headsep=1em]{geometry}

\usepackage[spanish]{babel}
\spanishlcroman
\usepackage[utf8]{inputenc}
\usepackage[T1]{fontenc}
\usepackage{lmodern}
\usepackage[letterspace=25]{microtype}

\usepackage{amsmath,amssymb,amsthm}
\usepackage{bm}
\usepackage{cases}
\usepackage{xfrac}
\usepackage{color}
\usepackage{framed}
\usepackage{fancyhdr}

\usepackage{enumitem}
\setlist[enumerate,1]{label=(\alph*)}
\setlist[enumerate,2]{label=\roman*.}

\usepackage{float}
\usepackage[center]{caption}

\usepackage{cite}

\newcommand\myauthor{Franco Frizzo}
\newcommand\mytitle{Tesis de licenciatura}
\newcommand\mydate{\today}

\usepackage[pdfauthor={\myauthor},
	    pdftitle={\mytitle},
	    hidelinks]{hyperref}

\setlength{\parskip}{.5em}
\renewcommand{\baselinestretch}{1.05}
\setlength{\headsep}{1.7em}
\setlist[enumerate]{itemsep=.1em, topsep=0em}
\setlist[itemize]{itemsep=.1em, topsep=0em}

\pagestyle{fancy}
\rhead{\MakeUppercase{\footnotesize{\textls{\myauthor}}}}
\lhead{\MakeUppercase{\footnotesize{\textls{\mytitle}}}}

\DeclareTextFontCommand{\emph}{\bfseries}

\theoremstyle{plain}
\newtheorem{teo}{Teorema}
\newtheorem{prop}[teo]{Proposición}
\newtheorem{coro}[teo]{Corolario}
\newtheorem{lema}[teo]{Lema}

\theoremstyle{definition}
\newtheorem{defi}[teo]{Definición}

\theoremstyle{remark}
\newtheorem*{obs}{Observación}
\newtheorem*{demo}{Demostración}
\newtheorem*{demosketch}{Idea de la demostración}

\newcommand{\note}[1]{\textbf{\textcolor{red}{#1}}}

\newcommand{\alphabet}{\ensuremath{\mathcal{A}}}

\newcommand{\neck}[1]{\left\vert#1\right\vert}
\newcommand{\substr}[2]{_{[#1\,..\,#2]}}

\newcommand{\mat}[1]{\mathbf{#1}}

\newcommand{\Succ}[1]{\ensuremath{\text{S}_{#1}}}
\newcommand{\Pred}[1]{\ensuremath{\text{P}_{#1}}}

\newcommand{\DB}[1]{\ensuremath{\text{DB}_{#1}}}
\newcommand{\M}[2]{\ensuremath{\text{M}_{#1}^{#2}}}
\newcommand{\NM}[2]{\ensuremath{\text{NM}_{#1}^{#2}}}
\newcommand{\Pf}[2]{\ensuremath{\text{P}_{#1}^{#2}}}
\newcommand{\NPf}[2]{\ensuremath{\text{NP}_{#1}^{#2}}}


\begin{document}

\title{\mytitle}
\author{\myauthor}
\date{\mydate}

\maketitle
\tableofcontents

\section{Introducción}

\section{Preliminares}

Acá podemos mencionar la fórmula cerrada para contar secuencias maravillosas.

\section{Resultados}

Teorema 1: para $n$, $m = 2^{n-1}$ existen las maravillosas anidadas que no son perfectas anidadas.

Teorema deseado: para $n = m$ existen las maravillosas anidadas que no son perfectas anidadas.

Teorema: para $n, m$ con $n > 2m$ no existen secuencias maravillosas anidadas.

Problema: determinar existencia para otros valores de $n, m$.

\subsection{Secuencias autosimilares}

Esta subsección tiene por objeto dar un método de construcción de secuencias maravillosas anidadas
para $n$, $m = 2^{n - 1}$. A través de este método
de construcción queda demostrado el Teorema 1.

	[Copiar ordenando y limpiando lo que está en el documento del logbook].

Mencionar la relación entre las cantidades de secuencias autosimilares y las cantidades de maravillosas.

\subsection{Otra construcción de secuencias $(n,m)$-maravillosas anidadas con $m = 2^{n - 1}$}

\begin{lema}
	Sean $x, y$ dos secuencias $(n - 1, m)$-maravillosas anidadas, que además representan ciclos $\frac{m}{2}$-hamiltonianos en el grafo de De Bruijn de orden $n$. Si se cumple que [la condición de pares cruzados para $n$ sobre los extremos de $x$ e $y$], entonces la concatenación
	$xy$ es una secuencia $(n, m)$-maravillosa anidada.
\end{lema}

\begin{proof}
	Demostrar este lema.
\end{proof}

\begin{prop}
	La reversa de este lema es falsa.
\end{prop}

\begin{proof}
	Damos un contraejemplo.
\end{proof}

\section{Estadísticas sobre secuencias maravillosas}

Cantidad de maravillosas anidadas para pares $n, m$. Comparando con cantidad de perfectas anidadas.

Cantidad de secuencias autosimilares para cada $n, m$.

Cantidad de secuencias que se originan pegando De Bruijn para cada $n, m$.

Secuencias que surgen por rotar la mitad derecha de secuencias perfectas.

Secuencias que surgen por rotar la mitad izquierda de las anteriores (en las que rotamos la parte derecha).

	[De las cantidades anteriores puede salir un nuevo método de construcción].

\bibliography{references}{}
\bibliographystyle{plain}

\end{document}
