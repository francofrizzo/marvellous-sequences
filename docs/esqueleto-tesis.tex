\subsection{Secuencias de De Bruijn}

Una secuencia de De Bruijn de orden $n$ es aquella en la cual
($s \in \DB{n}$) si todos los patrones $w \in \alphabet^n$ (de longitud $n$),
aparecen exactamente una vez en $\neck{s}$. En tal caso $\neck{s}$ es un
collar de De Bruijn.

Acá podemos mencionar la fórmula cerrada para contar secuencias maravillosas.

\section{Resultados}

Teorema 1: para $n$, $m = 2^{n-1}$ existen las maravillosas anidadas que no son perfectas anidadas.

Teorema: para $n, m$ con $n > 2m$ no existen secuencias maravillosas anidadas.

Problema: determinar existencia para otros valores de $n, m$.

\subsection{Secuencias autosimilares}

Esta subsección tiene por objeto dar un método de construcción de secuencia
maravillosas anidadas
para $n$, $m = 2^{n - 1}$. A través de este método
de construcción queda demostrado el Teorema 1.

  [Copiar ordenando y limpiando lo que está en el documento del logbook].

Mencionar la relación entre las cantidades de secuencias autosimilares y las cantidades de maravillosas.

\subsection{Otra construcción de secuencias $(n,m)$-maravillosas anidadas con $m = 2^{n - 1}$}

\begin{lema}
  Sean $x, y$ dos secuencias $(n - 1, m)$-maravillosas anidadas, que además representan ciclos $\frac{m}{2}$-hamiltonianos en el grafo de De Bruijn de orden $n$. Si se cumple que [la condición de pares cruzados para $n$ sobre los extremos de $x$ e $y$], entonces la concatenación
  $xy$ es una secuencia $(n, m)$-maravillosa anidada.
\end{lema}

\begin{proof}
  Demostrar este lema.
\end{proof}

\begin{prop}
  La reversa de este lema es falsa.
\end{prop}

\begin{proof}
  Damos un contraejemplo.
\end{proof}

\section{Estadísticas sobre secuencias maravillosas}

Cantidad de maravillosas anidadas para pares $n, m$. Comparando con cantidad de perfectas anidadas.

Cantidad de secuencias autosimilares para cada $n, m$.

Cantidad de secuencias que se originan pegando De Bruijn para cada $n, m$.

Conjetura: para $n = m$ existen las maravillosas anidadas que no son perfectas anidadas.

Secuencias que surgen por rotar la mitad derecha de secuencias perfectas.

Secuencias que surgen por rotar la mitad izquierda de las anteriores (en las que rotamos la parte derecha).

  [De las cantidades anteriores puede salir un nuevo método de construcción].
