%\begin{center}
%\large \bf \runtitulo
%\end{center}
%\vspace{1cm}
\chapter*{\runtitulo}

Consideremos un alfabeto finito de $b$ símbolos; las \emph{secuencias maravillosas}
de orden $(n,m)$ son secuencias de símbolos tomados de este alfabeto
tales que, al ser miradas circularmente, todas las
secuencias de longitud $n$ aparecen exactamente $m$ veces.
Las secuencias \emph{maravillosas anidadas} de orden $(n,m)$ son secuencias
maravillosas que son a su vez la concatenación de $b$ secuencias
maravillosas anidadas de orden $(n-1, m)$, salvo que $n = 1$.

Se sabe que siempre que $n$ es menor o igual que $m$ existen las secuencias
maravillosas anidadas que cumplen que todas las secuencias de longitud $n$
ocurren en \emph{distintas posiciones} módulo $m$.
¿Es necesaria esta condición? Dicho de otro modo, ¿aparecen \emph{nuevas} secuencias
maravillosas anidadas de orden $(n,m)$ si se elimina la restricción de que las
ocurrencias sean en distintas posiciones módulo $m$?

En esta tesis demostramos que para toda pareja $n, m$ con $m$ exponencial con
respecto a $n$ la respuesta es afirmativa, y presentamos un método de construcción.
Además, para un alfabeto de dos símbolos, demostramos si $n$ es mayor que $2m$
no existen tales secuencias.
Conjeturamos que, para todo $n$ menor o igual que $m + 1$, las secuencias
existen y presentamos algunos ejemplos.

\bigskip

% \noindent\textbf{Palabras clave:} ...
