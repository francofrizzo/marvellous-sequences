%\begin{center}
%\large \bf \runtitulo
%\end{center}
%\vspace{1cm}
\chapter*{\runtitulo}

Consideremos un alfabeto finito de $b$ símbolos; las \emph{secuencias maravillosas}
de orden $(n,m)$ son secuencias de símbolos tomados de este alfabeto
tales que, al ser miradas circularmente, todas las
palabras de longitud $n$ aparecen exactamente $m$ veces.
Las secuencias \emph{maravillosas anidadas} de orden $(n,m)$ son secuencias
maravillosas que, si $n > 1$, son a su vez la concatenación de $b$ secuencias
maravillosas anidadas de orden $(n-1, m)$.

Se sabe que para todos $n$ y $m$ tales que $n \leq m$ existen las secuencias
maravillosas anidadas que cumplen que todas las secuencias de longitud $n$
ocurren en \emph{distintas posiciones} módulo $m$.
¿Es necesaria esta condición? Dicho de otro modo, ¿aparecen \emph{nuevas} secuencias
maravillosas anidadas de orden $(n,m)$ si se elimina la restricción de que las
ocurrencias sean en distintas posiciones módulo $m$?

Partiendo de este interrogante, en esta tesis nos embarcamos en un estudio de
las secuencias maravillosas anidadas, arribando a distintas respuestas según los
valores de $m$ y $n$. Para los casos en que $n$ es del orden del logaritmo de
$m$ --en base al tamaño del alfabeto-- la respuesta es afirmativa, y presentamos
un método constructivo que permite obtener secuencias de estas características.
Para un alfabeto binario, demostramos que no existen tales secuencias en
los casos en que $n$ es mayor que $2m$.
En los casos restantes, realizamos un estudio computacional a partir
del cual extraemos ejemplos y conjeturas.

\bigskip

\noindent\textbf{Palabras clave:} ...