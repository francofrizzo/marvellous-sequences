%\begin{center}
%\large \bf \runtitle
%\end{center}
%\vspace{1cm}
\chapter*{\runtitle}

Consider an alphabet of $b$ symbols; a \emph{marvellous sequence} of
order $(n,m)$ is a sequence of symbols from this alphabet such that,
when looked at in a circular fashion, every possible word of length $n$ appears
exactly $m$ times.
\emph{Nested marvellous sequences} of order $(n,m)$ are marvellous
sequences that are also the concatenation of $b$ nested
marvellous sequences of order $(n-1,m)$, unless $n = 1$.

It is known that, whenever $n$ is less that or equal to $m$, there are marvellous
sequences for which every occurence of a word of length $n$ is in a different
position modulo $m$. Is this condition necessary? In other words,
are there any \emph{new} nested marvellous sequences of order $(n,m)$ that only
arise if the restriction that the occurences are in different positions modulo
$m$ is lifted?

In this thesis we prove that for every pair $n,m$ with $m$ exponential in $n$
the answer is affirmative, and present a constructive method.
Also, we show that it if $n$ is greater than $2m$ there are no such sequences.
We conjecture that sequences exist for every $n$ less than or equal to $m + 1$,
and we give several examples.
Finally, we prove that nested marvellous sequences can be used to construct
normal numbers ---in the Borel sense--- that converge to normality at the
fastest rate hitherto known.

\bigskip

\noindent\textbf{Keywords:} ...
