%\begin{center}
%\large \bf \runtitle
%\end{center}
%\vspace{1cm}
\chapter*{\runtitle}

Fix a finite alphabet of $b$ symbols; a \emph{marvellous sequence} of
order $(n,m)$ is a sequence of symbols taken from this alphabet such that,
when looked at in a circular fashion, every possible word of length $n$ appears
exactly $m$ times.
\emph{Nested marvellous sequences} are those marvellous sequences that, whenever
$n > 1$, are also the concatenation of $b$ nested marvellous sequences of order
$(n-1,m)$.

It is known that, for every $(n,m)$ such that $n \leq m$, there are marvellous
sequences in which every occurence of a word of length $n$ is in a different
position modulo $m$. Is this condition necessary? In other words,
are there any \emph{new} nested marvellous sequences of order $(n,m)$ that only
arise if the restriction that the occurences are in different positions modulo
$m$ is lifted?

Starting out from this question, in this thesis we engage in a study of nested
marvellous sequences, arriving at different answers depending of the values of
$n$ and $m$. In those cases where $n$ is of the order of the logarithm of $m$
--in the base of the alphabet size--, the response is affirmative, and we
present a constructive method that allows to obtain such sequences. On the other
hand, when $n$ is greater than $2m$, we show that such sequences do not exist.
For the remaining cases, we carry out a computational study from which we
draw examples and conjectures.

\bigskip

\noindent\textbf{Keywords:} ...